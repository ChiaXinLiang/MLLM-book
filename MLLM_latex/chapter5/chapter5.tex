\chapter{Applications of MLLMs in Vision-Language Tasks}

\chapterauthor{Li Ming}


\section{Image Captioning and VQA}
The integration of computer vision and natural language processing (NLP) has led to remarkable advancements in tasks like \textbf{Image Captioning} and \textbf{Visual Question Answering (VQA)}. Both fields aim to generate rich textual interpretations of visual data, requiring a deep understanding of visual elements and linguistic relationships. Historically, early models relied on hand-crafted features, but modern systems, particularly with the rise of \textbf{Multimodal Large Language Models (MLLMs)}, have revolutionized the performance and scope of these applications. Leveraging vast datasets and sophisticated architectures, MLLMs now generate captions and answer questions about images with unprecedented accuracy, making them integral to several real-world applications.

This paper surveys the progress of MLLMs in both image captioning and VQA, illustrating their technical underpinnings and applications.

\subsection{Image Captioning: Overview and Advances}
Image captioning refers to automatically generating textual descriptions for images by combining visual and linguistic processing. Early methods were limited to hand-crafted rules, but modern deep learning approaches, particularly through MLLMs, have dramatically improved performance. By training on large image-text datasets like MSCOCO and Flickr, MLLMs can generate rich and contextually accurate captions, significantly outperforming older methods \cite{icmeta2020m2transformer, icmeta2019oscar, icmeta2020densecap}.

Key advancements include techniques like:
\begin{itemize}
    \item \textbf{OSCAR (Object-Semantics Aligned Pre-training)}: Enhances captioning by aligning object tags with textual descriptions during pre-training. This leads to better object recognition and semantically rich captions \cite{icmeta2019oscar}.
    \item \textbf{VIVO (Visual Vocabulary Pre-training)}: Introduces a vocabulary of visual concepts, allowing models to caption novel objects unseen during training, crucial for real-world applications \cite{icmeta2019vivo}.
    \item \textbf{Dense Captioning}: A novel approach that generates region-specific captions for different parts of an image, useful in detailed image understanding and retrieval \cite{icmeta2020densecap}.
    \item \textbf{Generative Adversarial Networks (GANs)}: Applied to refine caption fluency and coherence by leveraging adversarial training \cite{icmeta2021gan}.
    \item \textbf{Meta-Learning Approaches}: Enable MLLMs to quickly adapt to new tasks with minimal data, improving generalization across various tasks, especially when training data is scarce \cite{icmeta2019meta}.
\end{itemize}

\subsection{Visual Question Answering (VQA): Overview and Advances}
VQA is an interdisciplinary field requiring a system to generate accurate answers to questions posed about images, combining both visual and textual reasoning \cite{vqa_survey2023}. Unlike image captioning, which describes an image holistically, VQA focuses on specific queries about an image's content.

MLLMs have made significant strides in VQA tasks by utilizing the same underlying multimodal architectures:
\begin{itemize}
    \item \textbf{MCAN (Multimodal Co-Attention Network)}: Uses co-attention mechanisms to fuse image and text features, resulting in improved understanding of image-question relationships \cite{mcan_vqa2019}.
    \item \textbf{Knowledge-Enhanced VQA Models}: Incorporate external knowledge graphs for commonsense reasoning, improving performance on complex VQA tasks \cite{lan2023improvingzeroshotvisualquestion}.
\end{itemize}

\subsection{Applications of Image Captioning and VQA}

The advancements in MLLMs have expanded their applications across diverse domains, offering solutions to various real-world problems by bridging the gap between visual data and language.

\begin{itemize}
    \item \textbf{Assistive Technologies} Both image captioning and VQA play critical roles in assistive technologies for the visually impaired. Image captioning systems convert visual scenes into real-time audio descriptions, helping users interpret their surroundings. For instance, applications like Microsoft Seeing AI use MLLM-generated captions to describe objects, people, and text in a user's environment \cite{icmeta2021assistive}. VQA enhances this by allowing users to ask specific questions about their environment, such as "What is the name on the sign?" or "Is there anyone near me?" This interactivity elevates the autonomy of visually impaired individuals, enabling more tailored assistance \cite{assistive_vqa2020}.
    \item \textbf{Autonomous Systems and Vehicles:} In autonomous vehicles, the ability to generate accurate captions and answer complex questions about the driving environment is critical for safety and decision-making. MLLMs generate captions describing road conditions, obstacles, pedestrians, and traffic signs, improving the vehicle's situational awareness \cite{icmeta2020autonomous}. VQA complements this by enabling the system to answer real-time queries like "Is there a stop sign ahead?" or "What is the speed limit?" enhancing the vehicle's capacity to navigate complex environments safely \cite{autonomous_vqa2019}.
    
    \item \textbf{Medical Imaging and Healthcare:} In medical imaging, MLLM-based captioning systems automatically generate diagnostic reports by interpreting X-rays, CT scans, and MRIs. This reduces the workload on radiologists and ensures faster report generation with high accuracy. Tools like CaptionHealth utilize these techniques to assist in real-time diagnostics \cite{icmeta2021medical}. VQA models enable clinicians to query specific aspects of medical images, such as "What abnormalities are present?" or "Is there evidence of pneumonia?" This leads to more interactive and accessible healthcare diagnostics, aiding in rapid decision-making \cite{med_vqa2019, healthcare_vqa2021}.
    
    \item \textbf{Content Moderation and Search Engines:} On social media platforms, MLLMs are used to generate captions that automatically tag images and flag inappropriate content, making content moderation more efficient. Platforms like Facebook and Instagram employ image captioning tools to scale moderation efforts by identifying objectionable or dangerous content from uploaded images \cite{icmeta2020content}. VQA further enhances content moderation by allowing moderators to query images directly, asking questions such as "Does this image contain violence?" or "Is there inappropriate text in the background?" This interactive capability allows for more nuanced and scalable moderation systems \cite{moderation_vqa2021}.
\end{itemize}


Multimodal Large Language Models have revolutionized the fields of image captioning and VQA, providing sophisticated tools that understand and interpret complex visual information through text. From assistive technologies to autonomous systems, healthcare to content moderation, the applications of MLLMs continue to grow, promising a future where machines can interpret and respond to visual data with human-like comprehension.

\section{Visual Storytelling and Scene Understanding}

\subsection{Introduction}

Multimodal Large Language Models (MLLMs) have profoundly impacted how AI handles tasks involving both visual and textual information. These models merge visual inputs with language processing, enabling systems to comprehend complex scenes and generate coherent narratives. This ability is especially crucial in fields like autonomous driving, interactive media, 3D modeling, and human-computer interaction \cite{vs2024li,vs2020hong}. This survey reviews recent advancements in visual storytelling and scene understanding, focusing on how MLLMs integrate visual and textual information to enable richer semantic understanding and narrative generation.

\subsection{Technologies for Visual Storytelling and Scene Understanding}

Visual storytelling, defined as generating coherent narratives from sequences of images or videos, has evolved from simpler object recognition methods to more sophisticated models that combine both visual semantics and contextual information. Early models often used scene graphs to capture relationships between objects but struggled with narrative coherence. The introduction of MLLMs has allowed for a more detailed interpretation of these relationships, integrating semantic layers for more nuanced storytelling.

For instance, the \textit{Multidimensional Semantic Augmented Network} provides a method for merging scene, object, and action semantics, enabling richer narrative construction \cite{vs2024li}. These models go beyond the static understanding of objects and scenes, instead focusing on how individual elements interact over time to form cohesive stories \cite{vs2024li}. In another example, \textit{Kosmos-1} leverages cross-modal knowledge transfer between vision and language, resulting in stories that are not only descriptive but also contextually relevant \cite{vs2024song}. By drawing on both visual and textual data, these models enable more fluent and complex storytelling compared to earlier systems.

Scene understanding, another key task, involves comprehending the spatial and relational structure of objects within a scene. Advanced models like \textit{Scene-LLM} integrate 3D visual data with textual descriptions, allowing for high-level reasoning about spatial relationships and object interactions. The use of hybrid 3D feature representations, which combine both global scene-level information and local object-centric details, enables models to reason about dynamic environments effectively \cite{vs2024rao}. This is particularly important in real-time applications, such as robotics and autonomous driving, where accurate scene understanding is critical to decision-making.

\subsection{Applications}

MLLMs are proving essential in several domains, where their ability to generate, understand, and manipulate multimodal data offers significant advantages.

In the \textit{entertainment industry}, MLLMs are used to automatically generate narratives for films, games, and other media. By analyzing sequences of images or video, these models can create storylines that evolve dynamically based on the characters' actions or environmental changes. For instance, games can now feature AI-driven storylines that adapt based on player decisions, creating an immersive and interactive experience \cite{vs2020parde}. Similarly, in content generation for streaming platforms, MLLMs are used to develop personalized narratives by analyzing viewer preferences and tailoring story elements accordingly \cite{vs2024song}.

In the realm of \textit{autonomous driving}, scene understanding is vital for real-time decision-making. MLLMs like \textit{OmniDrive} enhance autonomous systems' abilities to interpret 3D environments by accurately analyzing traffic situations, detecting potential hazards, and predicting the actions of other vehicles \cite{vs2024alvarez}. These models are trained on large datasets that include diverse driving scenarios, enabling them to generalize across different environments, including complex urban settings. By understanding the spatial and temporal relationships between objects, autonomous vehicles can make safer and more informed decisions on the road \cite{vs2024rao}.

For \textit{augmented reality (AR)} and \textit{interactive storytelling}, MLLMs offer the potential to create dynamic narratives that respond to user interactions with their environment. In AR applications, these models analyze the physical space around the user and generate stories or instructions that adapt based on real-world inputs. For example, in an AR-based learning environment, an MLLM might generate contextual stories or educational content based on the objects and scenes detected in a classroom or outdoor setting \cite{vs2019dey}. This ability to blend the physical and digital worlds creates highly personalized experiences that can be applied in education, entertainment, and marketing.

In \textit{robotics} and \textit{embodied AI}, MLLMs like \textit{Scene-LLM} improve robots' abilities to navigate and interact with their environments. Robots equipped with scene understanding models can perform tasks in complex, unstructured environments, such as warehouses or hospitals. These systems use 3D scene representations to understand their surroundings, make decisions about where to move, and interact with objects based on their spatial relationships \cite{vs2024rao}. In healthcare, for instance, robots might assist with patient care by navigating hospital environments and delivering medication based on their understanding of room layouts and equipment locations.

In \textit{content generation}, MLLMs have started to enable platforms that provide automatic captioning and summarization of images and videos. Applications such as digital marketing, social media content creation, and even journalism benefit from these models' ability to produce rich, engaging narratives from visual content \cite{vs2024zang}. This automated content creation reduces time and resources, while increasing personalization, allowing brands to engage more effectively with their audiences.

Future directions for MLLMs in these applications include improving real-time processing, enhancing model interpretability, and reducing the computational cost of deploying these large models in dynamic, resource-constrained environments. Addressing these challenges will be crucial for expanding MLLMs' use in more diverse, real-world scenarios \cite{vs2024yang,vs2023chen}.

\section{MLLM Applications in Content Creation and Editing}

\subsection{Introduction}

Multimodal Large Language Models (MLLMs) have significantly influenced the way content is created and edited, offering powerful tools for automating and enhancing tasks across various media formats. MLLMs excel in integrating text, image, video, and even audio data, allowing creators to generate, transform, and refine multimedia content. By leveraging these capabilities, MLLMs have become indispensable in fields such as multimedia storytelling, automated content generation, real-time editing, and collaborative creative workflows \cite{vs2024chang,vs2017lauer}. This section provides a detailed analysis of the technologies driving MLLMs and their specific applications in content creation and editing.

\subsection{Technologies Behind MLLM in Content Creation}

MLLMs are powered by several technological innovations, including transformers, Generative Adversarial Networks (GANs), and Vision-Language Models (VLMs). These technologies enable the models to understand and generate multimodal content, transforming industries such as entertainment, marketing, and journalism. 

The integration of GPT models for text generation with DALL-E and other image generation systems allows creators to produce entire multimedia pieces from simple prompts \cite{vs2024chang}. Advanced models like Video-LLaMA are pushing the boundaries by combining visual and textual inputs, enabling applications such as video editing and interactive storytelling. Self-supervised learning and multimodal training datasets enable these models to understand complex relationships between text, images, and videos, making them versatile tools for content creation \cite{vs2024song}.

\subsection{Applications in Content Creation and Editing}

The following are the key applications of MLLMs in the field of content creation and editing:

\begin{itemize}

\item \textbf{Multimodal Content Generation}: MLLMs allow artists and creators to generate images, text, and video content based on simple inputs. These models can interpret text prompts to create visual art, design characters, or craft environments for video games and films. This application is revolutionizing multimedia storytelling, where different types of content are seamlessly integrated to tell compelling narratives \cite{vs2024chang,vs2024schmidt}. MLLMs are also utilized in marketing and social media, where they analyze user preferences and trends to generate personalized content \cite{vs2020parde}.

\item \textbf{Real-Time Video and Image Editing}: MLLMs are revolutionizing video and image editing by enabling real-time modifications through natural language and visual inputs. Systems like ExpressEdit allow users to sketch over a video frame or provide verbal commands to alter specific elements in a scene \cite{vs2024tilekbay}. These models also support automated color correction, object tracking, and scene enhancements, making them invaluable in the film and advertising industries \cite{vs2024kubicek}.

\item \textbf{Automated Script and Article Writing}: MLLMs such as GPT-4 are highly effective in generating long-form text, including movie scripts, articles, and reports. By providing minimal input, users can rely on these models to produce well-structured and coherent content, freeing up time for more creative tasks \cite{vs2024eleftheriadis}. This application is especially valuable in journalism, where automated systems generate drafts that can be fine-tuned by editors \cite{vs2024anderson}.

\item \textbf{Collaborative Content Creation}: Collaborative content development tools powered by MLLMs allow teams to simultaneously work on different aspects of a project, from visual elements to text descriptions. Cloud-based platforms enable real-time collaboration, ensuring that all contributors are working on the most up-to-date version of the project \cite{vs2024santos}. In interactive design, tools like U-CREATE help creators develop augmented reality experiences and location-based services, streamlining the authoring process \cite{vs2024sauer}.

\item \textbf{Mobile Multimedia Editing}: Mobile applications integrating MLLMs have made content creation accessible to a broader audience. These apps allow users to edit and generate multimedia content on their devices using intuitive commands and gestures. By automating many technical aspects of editing, MLLM-powered mobile apps empower social media influencers and small businesses to create professional-grade content without requiring advanced editing skills \cite{vs2024jokela}.

\item \textbf{Content Repurposing and Multilingual Adaptation}: MLLMs are instrumental in repurposing content for different platforms and adapting it to various languages. Whether it's reformatting a blog post for social media or translating a promotional video into multiple languages, MLLMs maintain the original intent and adjust content to fit new formats and audiences \cite{vs2024obrenovic}. For global marketing campaigns, MLLMs can automate the localization of content, ensuring consistency across different regions while adapting to cultural nuances \cite{vs2024bateman}.

\item \textbf{Creative Personalization}: MLLMs are widely used for personalizing content based on user behavior and preferences. For example, personalized recommendations for entertainment platforms can be generated using MLLMs that analyze viewing habits, likes, and social trends \cite{vs2024bateman}. This application is particularly useful in e-commerce, where product descriptions and promotional materials are customized to reflect the tastes of individual users, improving engagement and conversion rates \cite{vs2024schmidt}.

\item \textbf{Dynamic Multimedia Creation}: MLLMs can dynamically generate multimedia presentations by combining textual, visual, and audio elements. This capability is invaluable in industries like education and training, where interactive and adaptive content is essential for engaging users. In educational technology, MLLMs assist in creating lesson plans, video tutorials, and interactive learning modules that adapt to the learner's pace and preferences \cite{vs2024anderson}.

\end{itemize}

MLLMs have ushered in a new era of content creation and editing by offering tools that automate and enhance creative processes. From multimodal content generation to collaborative editing, these models allow for the seamless integration of text, image, and video, transforming how multimedia content is produced. As MLLMs continue to evolve, we can expect further innovations in personalized content, real-time editing, and multilingual adaptation, opening new opportunities in industries ranging from entertainment to marketing \cite{vs2024chang,vs2024kubicek}.

\section{MLLM Applications in Cross-Modal Retrieval and Search}

\subsection{Introduction}

Cross-modal retrieval and search refers to retrieving relevant information from one modality (e.g., text) based on a query from another modality (e.g., image or audio). The growing availability of multimodal data, such as images, text, audio, and video, has made cross-modal retrieval a significant research area. Multimodal Large Language Models (MLLMs) have shown exceptional capability in addressing the complexities of cross-modal retrieval by understanding and relating different modalities. This section explores the technologies, methodologies, and applications of MLLMs in cross-modal retrieval and search, emphasizing recent advancements.

\subsection{Technological Foundations of Cross-Modal Retrieval}

MLLMs rely on a combination of multimodal transformers dual-encoder architectures, and self-supervised learning to manage the relationships between different types of data. For instance, Vision-Language Models (VLMs) such as CLIP and DALL-E have demonstrated their ability to map images to descriptive text, and vice versa, enabling efficient cross-modal retrieval. These models use contrastive learning to align visual and textual representations in a shared semantic space \cite{vs2024li,vs2015ranjan}. Other architectures, such as cross-attention mechanisms, help in precisely understanding the relationships between elements of different modalities.

Generative models like GPT-4V extend this capability by generating relevant outputs in one modality based on inputs from another. These models use large multimodal datasets to learn intricate correlations between visual, textual, and auditory data, enabling more nuanced retrieval results \cite{vs2024gomez}. 

\subsection{Applications of MLLMs in Cross-Modal Retrieval and Search}

\begin{itemize}

\item \textbf{Image-Text Retrieval} One of the most widespread applications of MLLMs in cross-modal retrieval is the search for images based on textual queries or vice versa. This has applications in areas such as e-commerce, where users can search for products using images or text descriptions. Models like CLIP map images and text into the same latent space, making it possible to retrieve images that semantically match the input text \cite{vs2024li}. Advanced models also allow retrieval of more abstract visual concepts, such as emotion or style, based on text queries.

\item \textbf{Video-Audio-Text Retrieval}: Cross-modal retrieval extends to the video and audio domains as well, with applications in multimedia search engines and content recommendation systems. For example, users can retrieve relevant video clips by providing a text description or even an audio snippet. Systems like Video-LLaMA and Speech2Text search can match video content with textual or spoken queries, allowing for accurate retrieval in large multimedia databases \cite{vs2014chen,vs2024gomez}. This has significant implications for media platforms, allowing users to discover video content based on both audio and text inputs.

\item \textbf{Generative Cross-Modal Retrieval}: Generative models are also pushing the boundaries of cross-modal retrieval by enabling systems to create new content from user queries. For example, DALL-E and similar models allow users to generate images based on detailed textual descriptions, while other systems are capable of generating music or video content from text-based prompts. This generative paradigm offers a new way of approaching content search and retrieval, where instead of finding existing content, users can generate what they need \cite{vs2024li,vs2019muller}.

\item \textbf{Multi-Lingual and Cross-Lingual Retrieval}: MLLMs also excel at cross-lingual retrieval, where text or speech in one language can be used to retrieve relevant content in another language. Systems trained on multilingual datasets, such as Speech2Text or multilingual transformers, enable retrieval of multimedia content across languages without needing parallel translation. This is particularly useful in global applications, where users may search in one language and receive content from another \cite{vs2024gomez,vs2024li}. For instance, a user could search using an English query and retrieve relevant French or Spanish media files, promoting a more inclusive digital ecosystem.

\item \textbf{Cross-Modal Music Retrieval}: Cross-modal retrieval extends beyond text, images, and video into music as well. Researchers have developed systems that enable users to find musical pieces based on descriptions of melodies, moods, or even visual stimuli such as album covers or sheet music \cite{vs2019muller}. These systems can enhance music recommendation platforms, allowing users to find music across various modalities, leading to more immersive listening experiences.

\item \textbf{Lecture Video Retrieval} Cross-modal retrieval systems are becoming increasingly important in educational technology. For example, models like the Multi-modal Language Model (MLM) have been applied to index and retrieve lecture videos based on both spoken content and text on slides \cite{vs2014chen}. This allows students to quickly find relevant portions of lecture videos by searching with either keywords or topics, greatly improving the efficiency of content retrieval in educational platforms.

\item \textbf{Content-Based Image Retrieval in Medical Domains} In the medical domain, cross-modal retrieval is being used to link textual descriptions (e.g., symptoms or diagnoses) with relevant medical images such as X-rays or MRIs. MLLMs trained on multimodal medical datasets can assist healthcare professionals in retrieving and analyzing medical data based on both text and images, improving diagnostic accuracy and speeding up research workflows \cite{vs2024jiang,vs2018dorfer}.

\item \textbf{Interactive Search in AR/VR}: Augmented reality (AR) and virtual reality (VR) applications are using cross-modal retrieval to enable more interactive and immersive experiences. Users can search for virtual objects, spaces, or experiences by describing them with words or gestures, and MLLMs process these inputs to retrieve or generate corresponding virtual environments or objects. This is particularly useful in gaming, training simulations, and virtual tourism \cite{vs2024li}.

\end{itemize}

Despite the progress in cross-modal retrieval, several challenges remain. One major challenge is the ability to handle domain-specific data such as medical or legal information, where cross-modal models must accurately interpret highly specialized data. Another challenge is the scalability of cross-modal retrieval in real-time systems, where the latency of retrieval responses is critical for applications like video streaming and e-commerce \cite{vs2024chang,vs2024gomez}. Lastly, improving the accuracy and contextual relevance of cross-modal retrieval results, especially in complex scenarios involving abstract concepts or multilingual inputs, remains an ongoing area of research.

Future directions include the enhancement of personalized retrieval systems that adapt to user preferences and the exploration of cross-modal reasoning capabilities, where models not only retrieve content but also infer and reason across modalities \cite{vs2024yin,vs2024palmagomez}. Additionally, integrating real-time data processing for AR and VR applications could open new possibilities in immersive and interactive cross-modal experiences.

MLLMs have revolutionized cross-modal retrieval and search by offering more accurate, scalable, and flexible solutions for handling multimodal data. From enabling image-text searches to creating generative content from user queries, MLLMs are expanding the boundaries of how we interact with multimedia content. As these models continue to evolve, they promise to play an even more central role in a variety of fields, including healthcare, education, entertainment, and beyond.

\section{MLLMs in Enhancing Accessibility for People with Disabilities}

\subsection{Introduction}
Multimodal Large Language Models (MLLMs) have emerged as powerful tools in various fields, including accessibility technologies. For individuals with disabilities, particularly those who are visually or hearing impaired, MLLMs present opportunities for significantly improving the quality of life through assistive technologies. By bridging modalities like text, image, audio, and video, these models can empower users with visual and hearing impairments, enabling smoother interactions with digital content and the real world \cite{vs2024yang,vs2024huang}. This section explores key technologies and their applications, highlighting how MLLMs have been adapted for accessibility purposes.

\subsection{Technological Foundations}
MLLMs, such as VIAssist and Kosmos-1, utilize cross-modal understanding to process inputs from different modalities, making them well-suited for accessibility solutions. Models like VIAssist are trained to recognize objects, generate descriptive text, and provide contextual answers to questions based on visual inputs \cite{vs2024yang}. These models combine techniques such as transformers, visual grounding, and natural language processing to create a seamless interaction layer between users and digital environments.

Moreover, cross-modal retrieval frameworks like those used in Kosmos-1 support real-time text generation from images or video streams, facilitating tasks such as captioning for the hearing impaired and object recognition for the visually impaired \cite{vs2024gomez}. The capability of MLLMs to handle both visual and linguistic data simultaneously underpins their effectiveness in accessibility technologies.

\subsection{Applications of MLLMs in Accessibility}

\begin{itemize}
    \item \textbf{Text-to-Speech and Speech-to-Text Systems}: MLLMs provide significant advancements in speech recognition and generation systems, which are crucial for the hearing impaired. Through models trained on vast speech and text datasets, these systems can accurately transcribe spoken content into text and vice versa. This technology has been particularly effective in creating subtitles in real-time for the hearing impaired, ensuring they can engage with live presentations, videos, and other media content \cite{vs2023chen,vs2024rao}. The integration of visual context into these models allows for more accurate transcription, making communication smoother in mixed-modal settings.

    \item \textbf{Visual Assistance for the Blind and Visually Impaired}: MLLMs are being leveraged to build systems that describe the surrounding environment to visually impaired users. Applications like VIAssist use cameras to capture images and then provide real-time narration or detailed descriptions of objects, people, or text present in the environment \cite{vs2024yang}. These models are trained to identify relevant aspects of scenes, eliminating unnecessary information and focusing on key details that assist the user in navigating or understanding their environment. This technology also extends to recognizing text from images, making documents and signs more accessible \cite{vs2024song}.

    \item \textbf{Object Detection and Recognition}: MLLMs enable sophisticated object recognition systems that benefit both visually impaired and hearing-impaired individuals. For instance, visually impaired users can rely on wearable devices equipped with cameras that, using MLLMs, can recognize objects and provide audio descriptions in real-time \cite{vs2024li}. For individuals with hearing impairments, visual object recognition can assist in lip-reading support or visual cues to better understand spoken content, enhancing their engagement in real-time conversations \cite{vs2020parde}.

    \item \textbf{Assistive Text Summarization and Captioning}: For individuals who are both visually and hearing impaired, real-time captioning and summarization of digital content are essential. MLLMs can generate captions for live events, meetings, and video content, making it easier for users to stay informed \cite{vs2020ramesh}. Furthermore, MLLMs are able to summarize large bodies of text, such as books, articles, and documents, converting them into audio or braille formats that are easier to consume for users with visual impairments. This reduces the cognitive load required to process large amounts of information, making technology more accessible to all \cite{vs2023gomez}.

    \item \textbf{Real-Time Sign Language Translation}: Another important application for the hearing impaired is sign language recognition and translation. MLLMs trained on multimodal datasets that include videos of sign language can now translate signs into text or spoken language, facilitating communication between sign language users and those unfamiliar with it \cite{vs2024wang}. These models employ visual data processing to recognize hand gestures, facial expressions, and body movements, interpreting them with a high degree of accuracy. This technology has the potential to drastically reduce communication barriers for hearing-impaired individuals \cite{vs2024huang}.

    \item \textbf{Personalized Accessibility Tools}: With the advancement of MLLMs, personalized accessibility tools have become a reality. These tools can learn individual user preferences and adapt their output accordingly, whether it's adjusting speech patterns, text formatting, or providing more contextual visual descriptions \cite{vs2024song}. Personalized accessibility not only improves the efficiency of these tools but also enhances the overall user experience by making the technology feel intuitive and responsive.

\end{itemize}

While MLLMs offer numerous advantages for accessibility, challenges remain. One of the main issues is ensuring that the models can generalize effectively across diverse environments and users. Models trained on specific datasets may struggle in real-world applications where the visual or auditory environment differs significantly from the training data. Additionally, the computational cost of running MLLMs in real-time for assistive technologies, particularly on mobile devices, remains high, limiting their widespread use \cite{vs2023huang}.

Future research will likely focus on improving the accuracy of real-time systems, expanding datasets to cover a broader range of use cases, and optimizing the models for lower-power devices. Moreover, advancements in multimodal reasoning, where models can infer meaning and context across diverse input types, will further enhance their effectiveness in accessibility technologies \cite{vs2023chen}.


Multimodal Large Language Models represent a transformative approach to enhancing accessibility for people with disabilities. From real-time sign language translation to visual assistance for the blind, these technologies offer the potential to break down communication barriers and empower individuals with disabilities. As research continues, the refinement of these models will bring about even greater benefits, making digital environments more inclusive and navigable for everyone \cite{vs2024yang,vs2024song,vs2024li}.


\bibliographystyle{plain}
\bibliography{chapter5/chap5_ref}

